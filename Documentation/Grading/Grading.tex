% Document class
\documentclass[a4paper, 12pt]{article}

% Font and Language settings
\usepackage{fontspec}
\usepackage{polyglossia}  % Multilingual support
\usepackage{csquotes}     % Quotation formatting
\usepackage{hyperref}     % For clickable links

% Language configuration
\setdefaultlanguage{english}   % Set default document language
\setotherlanguage{greek}       % Secondary language

% Main font configuration
\setmainfont{Linux Libertine O}[
    SmallCapsFeatures={Letters=SmallCaps}, % Enable small caps
    SmallCapsFont={Linux Libertine O},     % Font for small caps
    Ligatures=TeX,                         % Enable TeX ligatures
    Script=Latin                           % Specify script explicitly
]

% Sans-serif font configuration
\setsansfont{Linux Libertine O}[
    SmallCapsFont={Linux Libertine O},     % Font for small caps
    Ligatures=TeX,                         % Enable TeX ligatures
    Numbers=OldStyle,                      % Old style numbers
    Script=Latin                           % Specify script explicitly
]

% Monospace font
\setmonofont{Linux Libertine Mono O}

% Math support
\usepackage{amsmath}

% Graphics and drawing
\usepackage{graphicx}
\usepackage{tikz}
\usetikzlibrary{trees}  % TikZ tree diagrams


\usepackage{pgfplots}
\usetikzlibrary{intersections} % For intersection features
\usepgfplotslibrary{fillbetween} % For fill-between-paths functionality
\usetikzlibrary{shapes.geometric, positioning}

\tikzstyle{block} = [rectangle, draw, fill=blue!20, text centered, rounded corners, minimum height=3em]
\tikzstyle{line} = [draw, -latex] % Arrow style: a line with an arrowhead
\usepackage{amssymb}


\usepackage{tikz}
\usetikzlibrary{shapes.geometric, positioning}
\usepackage{multicol}
\usetikzlibrary{calc}

\newcommand{\includeimage}[2]{
    \begin{figure}[H]
        \centering
        \includegraphics[width=\linewidth]{#1}
        \if\relax\detokenize{#2}\relax
            % Do nothing if #2 is empty
        \else
            \caption{#2}
        \fi
    \end{figure}
}

% Code listings
\usepackage{listings}

% Table and figure placement
\usepackage{array}
\usepackage{float}

\begin{document}
\begin{center}
    \textbf{\Large CS359 - Web Programming}\\[0.5cm]
    \textbf{\Large Grading Report}\\[0.5cm]
    \textbf{\large Fall Semester 2024-2025}\\[1cm]
    \textbf{Spyridon Chrisovalantis Zervos (csd4878)}\\[0.2cm]
    \textbf{Rafail Drakakis (csd5310)}
\end{center}

\begin{table}[h!]
    \centering
    \begin{tabular}{|l|c|c|}
        \hline
        \textbf{Κατηγορία} & \textbf{Ποσοστό} & \textbf{Done (yes/no)} \\ \hline
        Σχεδίαση - Αναφορά & 15\% & yes \\ \hline
        Administrator & 25\% & yes \\ \hline
        Εγγεγραμμένος Χρήστης & 22\% & yes \\ \hline
        Εθελοντής Πυροσβέστης & 15\% & yes \\ \hline
        Επισκέπτης & 15\% & yes \\ \hline
        Σωστή Χρήση REST και AJAX & 4\% & yes \\ \hline
        Σωστή Χρήση Maps και Οπτικοποίησης & 4\% & yes \\ \hline
        Έξτρα Λειτουργικότητα (Bonus) & 8\% & yes \\ \hline
        \textbf{Σύνολο} & \textbf{108\%} & \\ \hline
    \end{tabular}
    \caption{Πίνακας Βαθμολόγησης}
\end{table}


\end{document}
